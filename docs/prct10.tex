\documentclass[spanish,11pt,a4paper]{article}
\usepackage[spanish]{babel}
\usepackage[utf8]{inputenc}

\begin{document}
 
\title{Número Pi} %Escribo un título
\author{Sandra Beatriz Jiménez Carballo} %Escribo el nombre
\date{11 de Abril de 2014} 
\maketitle

\begin{abstract}
 El número Pi~\cite{Lamport:LDP94} es la constante que relaciona el perímetro de una circunferencia con la amplitud de su diámetro: $\pi = L/D$.
 Este no es un número exacto sino que es de los llamados irracionales, ya que tiene infinitas cifras decimales.
\end{abstract}

\section{Historia}
Ya en la antigüedad, se insinuó que todos los círculos conservaban una estrecha dependencia entre el contorno y su radio pero tan sólo desde el
siglo XVII la correlación se convirtió en un dígito\cite{gibaldMLA:2009} y fue identificado con el nombre "Pi" (de periphereia, denominación que los griegos daban
al perímetro de un círculo).

Esta notación fue usada por primera vez en 1706 por el matemático galés William Jones y popularizada por el matemático Leonard Euler en su obra
"Introducción al cálculo infinitesimal" de 1748. Fue conocida anteriormente como constante de Ludoph (en honor al matemático Ludolph van Ceulen)
o como constante de Arquímedes (No se debe confundir con el número de Arquímedes).

\subsection{Euclides}
Euclides precisa en sus Elementos los pasos al límite necesarios e investiga un sistema consistente en doblar, al igual que Antiphon, el número de
lados de los polígonos regulares y en demostrar la convergencia del procedimiento.

\subsection{Arquímedes}
Arquímedes reúne y amplía estos resultados.\footnote{Trabajo sobre PI} Prueba que el área de un círculo es la mitad del producto de su radio por la circunferencia y que la
relación del perímetro al diámetro está comprendida entre 3,14084 y 3,14285.

\section{Enfoque matemático}
A pesar de tratarse de un número irracional continúa siendo averiguada la máxima cantidad posible de decimales. Los cincuenta primeros son:
$\pi = 3,14159265358979323846264338327950288419716939937510$ 

\subsection{Definiciones}
$\pi$ es:

\begin{itemize}
\item La relación entre la longitud de una circunferencia y su diámetro.
\item El área de un círculo unitario.
\item El menor número real x positivo tal que $\sin(x) = 0$
\end{itemize}

\subsection{Geometría}

\begin{itemize}
\item Longitud de la circunferencia de radio r: $ C = 2 \pi r $
\item Área del círculo de radio r: $ A = \pi r $
\end{itemize}

\bibliographystyle{plain}
\bibliography{prct10}

\end{document}
